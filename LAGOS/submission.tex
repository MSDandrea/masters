\documentclass[9pt]{./document-types/entcs} \usepackage{./document-types/entcsmacro}
\usepackage{graphicx}
\usepackage{dot2texi}
\sloppy
% The following is enclosed to allow easy detection of differences in
% ascii coding.
% Upper-case    A B C D E F G H I J K L M N O P Q R S T U V W X Y Z
% Lower-case    a b c d e f g h i j k l m n o p q r s t u v w x y z
% Digits        0 1 2 3 4 5 6 7 8 9
% Exclamation   !           Double quote "          Hash (number) #
% Dollar        $           Percent      %          Ampersand     &
% Acute accent  '           Left paren   (          Right paren   )
% Asterisk      *           Plus         +          Comma         ,
% Minus         -           Point        .          Solidus       /
% Colon         :           Semicolon    ;          Less than     <
% Equals        =3D           Greater than >          Question mark ?
% At            @           Left bracket [          Backslash     \
% Right bracket ]           Circumflex   ^          Underscore    _
% Grave accent  `           Left brace   {          Vertical bar  |
% Right brace   }           Tilde        ~

% A couple of exemplary definitions:


\newcommand{\Nat}{{\mathbb N}}
\newcommand{\Real}{{\mathbb R}}
\def\lastname{Please list Your Lastname Here}
\begin{document}
\begin{frontmatter}
  \title{An Example Paper} \author{My
    Name\thanksref{ALL}\thanksref{myemail}}
  \address{My Department\\ My University\\
    My City, My Country} \author{My Co-author\thanksref{coemail}}
  \address{My Co-author's Department\\My Co-author's University\\
    My Co-author's City, My Co-author's Country} \thanks[ALL]{Thanks
    to everyone who should be thanked} \thanks[myemail]{Email:
    \href{mailto:myuserid@mydept.myinst.myedu} {\texttt{\normalshape
        myuserid@mydept.myinst.myedu}}} \thanks[coemail]{Email:
    \href{mailto:couserid@codept.coinst.coedu} {\texttt{\normalshape
        couserid@codept.coinst.coedu}}}
\begin{abstract}
  This is a short example to show the basics of using the ENTCS style
  macro files.  Ample examples of how files should look may be found
  among the published volumes of the series at the ENTCS Home Page
  \texttt{http://www.elsevier.com/locate/entcs}.
\end{abstract}
\begin{keyword}
  Please list keywords from your paper here, separated by commas.
\end{keyword}
\end{frontmatter}





\section{Introduction}\label{intro}

\begin{dot2tex}[neato, options=-tmath]
graph G
{
        node [shape="circle"];

        a -- b [label = "-"];
        a -- c [label = "-"];
        {rank=same; "b" -- "c" [label="+"];}
}
\end{dot2tex} 

\section{Começou}

{\sc Coloring($r,\ell$)} can be easily solved in polynomial time when $(r,\ell) \in \{(0,1),(1,0),(1,1),(2,0)\}$.
Therefore, in order to completely classify the complexity when $r+\ell \leq 2$ remains to analyse the $(0,2)$-case.

	\begin{lemma}
{\sc Coloring($0,2$)} can be solved in polynomial time.
\end{lemma}
	\begin{proof}
$(0,2)$-graphs, also known as co-bipartite, is a subclass of perfect graphs~\cite{bollo98}, consequently its chromatic number
equals to its clique number, which can be found in polinomial time using a polynomial algorithm for {\sc Independent Set}
on bipartite graphs (see~\cite{konig31,zing12}).
	\end{proof}

As $(3,0)$-graphs (tripartite graphs) have chromatic number at most three, and to verify if the chromatic number of a graph $G$
is one or two is trivial, we know that {\sc Coloring($3,0$)} can be solved in polynomial time.
Next we consider $(4,0)$-graphs.

\begin{lemma}
{\sc Coloring($4,0$)} is NP-Complete.
\end{lemma}
\begin{proof}
It is well-known that any planar graph is $(4,0)$ (4-colorable)~\cite{appel77}, and X and Y provide an $O(n^2)$ algorithm to find a 4-coloring of a planar graph.
In addition, to determine whether a planar graph $G$ is 3-colorable is NP-complete~\cite{larry} (even when a $(4,0)$-partition of $G$ is also provided).
Therefore to determine the chromatic number of $(4,0)$-graphs is NP-complete.
\end{proof}



\section{Bibliographical references}\label{references}

\begin{thebibliography}{10}\label{bibliography}

\end{thebibliography}

\end{document}
